\documentclass[12pt]{article}
\usepackage[tagged, highstructure]{accessibility}
\usepackage[english]{babel}
\usepackage[utf8x]{inputenc}
\usepackage[T1]{fontenc}
\usepackage[margin=1in]{geometry}
\usepackage{scribe}
\usepackage{listings}
\usepackage{natbib,verbatim}
\usepackage{hyperref}
\hypersetup{
    colorlinks=true,
    linkcolor=blue,
    filecolor=magenta,      
    urlcolor=magenta,
    pdftitle={Course Syllabus},
    pdfauthor={Nisha Chandramoorthy},
    pdflang={en-US}
}

%\Scribe{Your Name}
\title{Syllabus-STAT/CAAM-37781}
\Lecturer{Nisha Chandramoorthy (nishac@uchicago.edu)}
\LectureNumber{Kernel methods: theory and computation}
\LectureDate{Jan 6, 2025}
\LectureTitle{}

\lstset{style=mystyle}

\begin{document}
\MakeScribeTop
Kernel functions measure similarity between data points. For tasks such as classification, regression, 
density estimation, generative modeling and so on, dependencies among the data points must be learned.
Algorithms for such tasks using kernel functions as a similarity measure on data space are called kernel methods. 
In this class, we study both the theory behind such methods and their computational aspects. 
\section{General information}
\begin{itemize}
	\item Two lectures per week, 2 homeworks, 1 final project and 1 short seminar (no exams).
	\item Class time and location: Mondays and Wednesdays, 3:00 pm -- 4:20 pm, Jones 303
	\item Office hours: by appointment
	\item Instructor email: nishac@uchicago.edu
\end{itemize}

\section{Grading information}

This is a seminar-style course with no exams. The grade will be determined by a final project (60\%), homeworks (20\%) and your seminar (20\%). 
\begin{itemize}
\item \textbf{Final project}: The final project has to be done individually, and the deliverables include a proposal, code, accompanying report and a 10-minute in-class presentation. A final project rubric and a set of guidelines will be posted on canvas before the proposal due date. All written material should be typed up and submitted on Gradescope. 
\item \textbf{Homeworks}: there will be 2 homework assignments (due dates on Canvas, spread out evenly through the semester before the final project) that will be theoretical, and often require numerical solutions. 
\item \textbf{Seminar}: each student should give a 20-minute presentation of a paper related to kernel methods (classical or recent). You are strongly encouraged to implement the numerical results and re-derive the theoretical results by yourself before you deliver your lecture. Following each lecture, there will be 5-8 minutes of Q\&A.
\end{itemize}


\section{Honor code}
You are always welcome to discuss with other students and use online resources, including AI assistants such as ChatGPT and Github CoPilot. After that, however, all your submitted work and your presentations should be your own. 
Please submit typed up homework solutions (handwritten solutions are often illegible and will not be graded) on Gradescope as a pdf, and do not copy directly from anywhere.

\section{Class site}
The materials from this class will be uploaded to Github as well as to Canvas. Check the websites for detailed lecture schedules, notes and announcements.
%%%%%%%%%%% end of doc
\end{document}
